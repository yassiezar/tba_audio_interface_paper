\documentclass{hcij_article}

\begin{document}

\articletitle{Experimental Analysis of a Spatialised Audio Interface for People with Visual Impairments}

\authorlist{%
  Jacobus C. Lock\textsuperscript{1}, Iain D. Gilchrist\textsuperscript{2}, Grzegorz Cielniak\textsuperscript{1} and Nicola Bellotto\textsuperscript{1}%
}

\affiliationlist{\textsuperscript{1}University of Lincoln, Lincoln, LN6 7TS, UK; \textsuperscript{2}University of Bristol, Bristol, BS8 1TH, UK%
}

\titleauthorminibios

\miniauthorbio{Jacobus C. Lock}{jaycee.lock@gmail.com}{jclock.co.uk}{is a research assistant and PhD student at the University of Lincoln. His research interests are on intelligent systems and mobile computing.}

\miniauthorbio{Iain D. Gilchrist}{i.d.gilchrist@bristol.ac.uk}{}{is a professor at the University of Bristol's School of Psychological Science. His major academic focus is on how humans gather information about their visual environment, in particular how and why our eyes move to sample the world.}

\miniauthorbio{Grzegorz Cielniak}{gcielniak@lincoln.ac.uk}{}{is an associated professor at the University of Lincoln's School of Computer Science. His research interests include mobile robotics, machine perception, AI and in particular applications in agricultural robotics and food technology.}

\miniauthorbio{Nicola Bellotto}{nbellotto@lincoln.ac.uk}{}{is an associated professor at the University of Lincoln's School of Computer Science. His research focus is on machine perception of humans for robotics and intelligent systems, from low-level detection and tracking, to high-level behaviour understanding and reasoning.}

\titlenotes

%\titlebackground{This article is based on...}
\titlefunding{This project is partly funded and supported by the Google Winter Award (2016).}
%\titlesupplementarydata{Supplemental data can be found...}

\titlehcieditorialrecord{First received on \emph{date}. Revisions received on \emph{date}, \emph{date}, and \emph{date}. Accepted by \emph{action-editor-name}. Final manuscript received on \emph{date}.}

\end{document}
